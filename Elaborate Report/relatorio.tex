\documentclass[12pt,titlepage]{article}

\usepackage[brazil]{babel}
\usepackage{setspace}
\usepackage{graphicx}
\usepackage[utf8]{inputenc} %use UTF8 encoding
\usepackage{listings}
\usepackage{xcolor}
\usepackage{caption}
\usepackage{minted}
\usepackage{multicol}
\usepackage{subcaption}
\usepackage{titlepic}
\usepackage{fancyhdr} %%Fancy headings



\fancypagestyle{front}{

  \fancyhf{}
  \renewcommand{\headrulewidth}{0pt}
  \fancyhead[L]{\raisebox{-0.5\height}{\includegraphics[scale=0.3]{logo.png}}
	}
\fancyhead[R]{
\raisebox{-0.5\height}{\includegraphics[scale=0.15]{poli.jpeg}}
}
}


\colorlet{codegray}{gray!20}

\DeclareCaptionFont{white}{\color{white}}
\DeclareCaptionFormat{listing}{\colorbox{gray}{\parbox{\textwidth}{#1#2#3}}}
\captionsetup[lstlisting]{format=listing,labelfont=white,textfont=white}

\lstset{ % General setup for the package
	language=Prolog,
	basicstyle=\small\sffamily,
	numbers=left,
 	numberstyle=\tiny,
	frame=tb,
	tabsize=4,
	columns=fixed,
	showstringspaces=false,
	showtabs=false,
	keepspaces,
	commentstyle=\color{red},
	keywordstyle=\color{blue}
}

\lstdefinestyle{myPrologstyle}
{
    language=Prolog,
    basicstyle = \ttfamily\color{blue},
    moredelim = [s][\color{black}]{(}{)},
    literate =
        {:-}{{\textcolor{black}{:-}}}2
        {,}{{\textcolor{black}{,}}}1
        {.}{{\textcolor{black}{.}}}1
}


\title{Modelagem e simulação de sistemas de planejamento automático usando inteligência artificial}
\date{20/01/2018}
\author{Leonardo Calasans}
%\titlepic{\includegraphics{logo.png}}


%\graphicspath{/Desktop/Poli/Iniciação Científica/Relatorio/Images}
\begin{document}

\begin{titlepage}

\thispagestyle{front}


\Large


%\vspace{-10pt}
%\begin{minipage}{0.3\textwidth}
%\begin{center}
%\end{center}
%\end{minipage}%
%\hfill
%\begin{minipage}{0.2\textwidth}
%\begin{center}
%\begin{figure}[H]
%\includegraphics[width=0.5\linewidth, height=0.5\linewidth]{logoecole.jpg}
%\end{figure}
%\end{center}
%\end{minipage}%

%\\
%\vspace*{1cm}

\begin{center}
%\vspace*{3cm}
\textsc{Relatório de andamento\\[0.5\baselineskip]
Escola Politécnica\\[0.5\baselineskip]
}

\vspace{3cm}

%\LARGE
\huge{\textsc{Modelagem e simulação de sistemas de planejamento automático usando inteligência artificial}}\\[0.5\baselineskip]
Leonardo Calasans\\
{\normalsize \textsc{}}\\



\vspace{1.5cm}
\LARGE{\textsc{Orientador:}}\\[0.1\baselineskip]
 \huge José Reinaldo Silva\\[0.4\baselineskip]
\normalsize \par \textsc{Departamento de Engenharia Mecatrônica / Universidade de São Paulo \\
}


\vfill
\LARGE{\textsc{20 de Janeiro de 2018}}\\

\end{center}

\end{titlepage}


\pagenumbering{gobble}
%\maketitle
\newpage
\doublespacing
\tableofcontents
\newpage
\pagenumbering{arabic}


\begin{abstract}
		\paragraph{}
		A área de inteligência artificial é de grande importância para o desenvolvimento científico e tecnológico. Assim, se faz imprescindível obter um conhecimento mais aprofundado e prático nessa área ainda na graduação.
		\paragraph{}
		Para este trabalho, foram  estudados conceitos de planejamento de inteligencia artificial, tais como tipos de planejadores, algoritmos de busca, heurísticas, e métodos formais de descrição de problemas. Esses conhecimentos foram então aplicados no desenvolvimento de um planejador capaz de resolver um problema canônico da inteligencia artificial.
		\paragraph{}
		Não bastando apenas obter um plano e validá-lo em ferramentas e planejadores já existentes, buscou-se também uma maneira de simular esse plano, seja em ambiente virtual ou real, utilizando um framework usado mundialmente para controle e construção de robôs.

\end{abstract}
\newpage

\section{Introdução}

	\paragraph{}
	Sistemas de planejamento de inteligência artificial é assunto de muitos trabalhos científicos e base para diversas aplicações tecnológicas de inteligencia artificial. Dessa maneira, aprender conceitos envolvidos na construção desses planejadores é de grande interesse para formação acadêmica.
	\paragraph{}
	Alem disso, o custo de aplicar os planos obtidos às vezes é muito alto, sendo interessante possuir um ambiente virtual para simulação e validação desses planos. Neste trabalho, usa-se ROS para simulação e propõe-se o uso do Gazebo como ambiente virtual de simulação. 



\section{Modelagem de sistema de planejamento}
\subsection{Solução do problema dos blocos}
\subsubsection{O problema}

\paragraph{}

O "mundo dos blocos" consiste em um conjunto de uma superfície e uma certa quantidade de blocos que devem ser empilhados de determinada maneira por um agente manipulador com algumas restrições : a) só é possível manipular um bloco por vez e b) não é possível manipular um bloco se há outro em cima dele.\cite{intart}

\paragraph{}

Dependendo da disposição inicial e final desejada, o problema torna-se não-trivial devvido à chamada "Anomalida de Sussman" \cite{modernapp}. Uma dessas disposições é mostrada na figura 1.

\begin{figure}[h!]
  \centering
  \includegraphics[width=\linewidth]{sussman.png}
  \caption{Exemplo de disposição que passa pela Anomalia de Sussman}
  \label{fig:figura 1}
\end{figure}

\paragraph{}

Nesses casos, planejadores que simplemente resolvem os sub-objetivos isoladamente não são capazes de resolver o problema, pois ao resolver um dos sub-objetivos, um dos demais, completado anteriormente, é desfeito.
\paragraph{}

No exemplo da figura 1, o sub-objetivo de ter A em cima de B é desfeito quando se vai cumprir o sub-objetivo de ter B em cima de C.

\subsubsection{Solução em Prolog}

Para resolver o problema da Anomalia de Sussman, foi desenvolvido um programa, em lingaugem Prolog \cite{prolog} e utilizado um sistema de descrição formal de problemas de inteligência arficial denominado STRIPS \cite{intart}.
\paragraph{}

	Uma instância STRIPS é uma quádrupla (P,O,I,G) em que:

  \begin{itemize}
    \item 		“P” representa o conjunto de condições inerentes ao problema analisado
    \item 		“O” representa o conjunto de ações possíveis nesse problema. Cada ação é representada pelas suas pré-condições e pós-condições, ou seja, condições necessárias para que a ação possa ser realizada e as condições que a ação tornam verdadeiras.
    \item 		“I” representa o estado inicial do problema
    \item     “G” representa o estado final do problema

  \end{itemize}
	\paragraph{}

  Parte de cada um dos itens que compõe a instância STRIPS desse problema são mostradas nas linhas de código abaixo.

\singlespacing
\begin{small}

\begin{minted}[bgcolor = codegray]{Prolog}
block(a).
block(b).
block(c).
clear(hand).
clear(table).

\end{minted}

\captionof{lstlisting}{Conjunto de condições inerentes ao problema analisado ("P")}

\bigskip

\begin{minted}[bgcolor = codegray]{Prolog}
putdown(X,A) :-
      %precond
	block(X),
	block(A),
	held(X),
	clear(A),
	A \== X,
      %effects
        assert(clear(hand)),
	assert(on(X,A)),
	assert(clear(X)),
	retract(clear(A)),
	retract(held(X)),
	assert(move(putdown(X,A))).
\end{minted}
\captionof{lstlisting}{Definição de uma das ações do conjunto "O"}

\bigskip

\begin{minted}[bgcolor = codegray]{Prolog}
%Initializer

init([]) :-
	true,
	init_clear([a,b,c]).


init([H|L]) :-
  initialize(H),
	init(L).

initialize(on(X,Y)) :-
	assert(on(X,Y)).


init_clear([]) :-
	true.

init_clear([H|L]) :-
	(initialize_clear(H); true),
	init_clear(L).

initialize_clear(X) :-
	not(on(_,X)),
	assert(clear(X)).
\end{minted}
\captionof{lstlisting}{Inicialização do problema e construção do conjunto "I"}

\bigskip

\begin{minted}[bgcolor = codegray]{Prolog}
%Solving

solve([H|R],[G|L]) :-
  init([H|R]),
  solve_all([G|L], [G|L]).

solve_all([G|L],Glist) :-
	G,
	solve_all(L,Glist).

solve_all([G|_], Glist) :-
	do(G),
	solve_all(Glist,Glist).

solve_all([],_).


%Doing

do(G) :-
	G.

do(on(A,B)):-
	force_putdown(A,B).
do(on(A,table)):-
	force_put_on_table(A).
\end{minted}
\captionof{lstlisting}{Resolução do problema com "G" dado pelo usuário}


\doublespacing
\end{small}

\bigskip


 Assim, o programa recebe como entrada duas listas, representando os estados inicial e final, e retorna a lista de movimentos necessários :

 \singlespacing
 \begin{small}


 	\begin{minted}[bgcolor = codegray]{Prolog}

?- solve([on(a,table),on(b,table),on(c,a)],[on(a,b),on(b,c),on(c,table)]).
 true .

 ?- listing(move).
 :- dynamic move/1.

 move(pickup(c)).
 move(put_on_table(c)).
 move(pickup(a)).
 move(putdown(a, b)).
 move(pickup(a)).
 move(put_on_table(a)).
 move(pickup(b)).
 move(putdown(b, c)).
 move(pickup(a)).
 move(putdown(a, b)).

 true.

 	\end{minted}
	\captionof{lstlisting}{Resolução do exemplo da figura 1}


\doublespacing
\end{small}

\bigskip

\subsubsection{Solução usando o itSIMPLE 4.0}

  Após a resolução do problema do mundo dos blocos em Prolog, houve o aprendizado sobre Redes de Petri,UML \cite{uml} e sobre o funcionamento da ferramenta itSIMPLE \cite{itsimple2005} \cite{itsimple3.0} \cite{itsimple2.0} \cite{itsimple4.0}, que possibilita modelar problemas de maneira mais eficiente , obter planos de solução e Redes de Petri representando-os.
	\paragraph{}

	A figura 2 mostra a modelagem do problema da figura 1 no itSIMPLE e o plano obtido utilizando o planejador Metric-FF.

\newpage
	\begin{figure}[H]
	  \centering
	  \begin{subfigure}[b]{0.3\linewidth}
	    \includegraphics[width=\linewidth, height=5 cm]{init.png}
	     \caption{Estado inicial}
	  \end{subfigure}
	  \begin{subfigure}[b]{0.3\linewidth}
	    \includegraphics[width=\linewidth, height=7 cm]{goal.png}
	    \caption{Estado final}
	  \end{subfigure}
		\begin{subfigure}[b]{0.8\linewidth}
			\includegraphics[width=\linewidth]{pn.png}
			\caption{Rede de Petri do problema}
		\end{subfigure}
		\begin{subfigure}[b]{0.5\linewidth}
			\includegraphics[width=\linewidth]{plan.png}
			\caption{Plano obtido}
		\end{subfigure}
		\caption{Modelagem no itSIMPLE 4.0}

	\end{figure}

\section{Simulação do sistema de planejamento}

	Após a modelagem e validação de um sistema inteligente capaz de resolver o problema do mundo dos blocos, passou-se ao estudo de ferramentas para criar simulações dos planos obtidos. Decidiu-se pelo uso do ROS e Gazebo.

\subsection{Sistema em ROS}

	ROS (Robot Operating System) é um framework que conta com um conjunto de bibliotecas open source em Python ou C++ para construir e controlar robôs. Sistemas em ROS contam com a "abstração de hardware", que significa que um sistema ROS feito para um robô funciona em qualquer outro também construído em ROS. O funcionamento básico desses sistemas se dá pelo contato entre os diversos "nós" que o compõe.\cite{ros}
	\paragraph{}

	Em parceria com o grupo RoboFEI da Faculdade de Engenharia Industrial, obteve-se acesso a um robô manipulador. Para programar o robô a fim de que ele fosse capaz de executar um plano para o mundo dos blocos, foi feito, em Python, um sistema no ROS chamado de judith\_dlab.
	\paragraph{}

	O panorama do Judith-Dlab é mostrado na figura 3, e conta com 3 nós:

	\begin{itemize}
		\item Interpreter: Nó que recebe o plano e o estado inicial diretamente do planejador em Prolog (visto na seção 2.2) e transforma essas informações em comandos entendíveis pelo robô.
		\item Environment : Nó que administra todas as ações tomadas e atualiza o estado de acordo.
		\item Communicator : Nó que recebe as informações processadas e envia ao robô as coordenadas para onde o manipulador deve se mover.
	\end{itemize}

	\begin{figure}[H]
		\centering
		\includegraphics[width=8cm, height = 8cm, keepaspectratio]{judith_dlab.png}
		\caption{Panorama do sistema judith\_dlab}
	\end{figure}

	\paragraph{}

	Foram realizadas simulações no computador do funcionamento do judith\_dlab, uma delas é mostrada na figura 4 (por questão de espaço, é mostrado um caso mais simples que o da figura 1)

	\begin{figure}[h!]
		\centering
		\includegraphics[width=10cm,height=10cm, keepaspectratio]{ros.png}
		\caption{Funcionamento do sistema}
	\end{figure}

	\paragraph{}

	O "matricial state", mostrado na figura é um estado intermediário utilizado internamente no Interpreter e usado para verificar possíveis erros. As três linhas de [INFO] são as coordenadas (x,y,z) que são enviadas ao robô a cada interação pelo Communicator.

	\paragraph{}

	Por problemas de hardware do manipulador da FEI, não foi possível implementar o sistema ROS neste robô até o presente momento.

\section{Continuidade do projeto}
\paragraph{}

	A fim de evitar problemas com hardware externo, como o ocorrido com o manipulador da FEI, decidiu-se adotar um ambiente virtual de simulação e construção de robôs integrado com o ROS, o Gazebo.
	\paragraph{}

	Algumas vantagens do Gazebo são suas engines de física de alta performance, gráficos 3D, a possibildade de rodar simulações remotamente, e as diversas bibliotecas de integração com o ROS.
	\paragraph{}

	No presente momento, se faz o estudo sobre tais bibliotecas, entre elas a "tf2", que permite controlar conjuntamente diversos sistemas de coordenadas, "URDF", que é um formato universal de descrição e construção de robôs, e a "ROSPlan", que permite integrar planejadores mais sofisticados com o ROS e o Gazebo.
	\paragraph{}

	O objetivo imediato é construir no Gazebo um robô manipulador, um ambiente virtual do "mundo dos blocos", e realizar a simulação desejada dentro do Gazebo. Após isso, o objetivo é construir no Gazebo um ambiente de simulação multipropósito, capaz de se integrar com a ferramenta itSIMPLE e simular ambientes e problemas mais complexos.

\section{Conclusão}

	\paragraph{}
	O plano obtido no planejador desenvolvido em Prolog, apesar de apresentar um passo a mais que aquele obtido no Metric-FF, alcança corretamente o objetivo. Dessa forma, sabe-se que o planejador desenvolvido está correto e validado por ferramentas mais avançadas.
	\paragraph{}
	O sistema de simulação em ROS também alcança os objetivos corretamente em um ambiente de simulação, apesar de ainda não ter sido validado em um robô real ou virtual.
	\paragraph{}
	A realização deste trabalho trouxe diversos conhecimentos importantes que servirão de base em futuros projetos, entre eles o aprendizado de conceitos de inteligência artificial, de controle e construção de robôs utilizando ROS, e simulação usando Gazebo. Todos esses conhecimentos são universais e podem ser aplicados em diversas áreas da robótica e da inteligência artificial.




\newpage

\nocite{}
\bibliography{relatorio}
\bibliographystyle{ieeetr}

\end{document}
